 
A indústria de petróleo e gás utiliza tubos flexíveis em suas linhas de produção devido a 
suas características estruturais de baixa rigidez à flexão e alta rigidez axial.
Os tubos flexíveis são compostos por várias camadas de materiais onde a 
camada mais interna dos tubos possui cavidades que são espaçadas entre si por uma distância constante.
As cavidades, na camada interna dos tubos flexíveis, são responsáveis pelas
características de flexibilidade dos tubos. A presença de cavidades nos tubos pode proporcionar alterações dinâmicas no
escoamento, como por exemplo, alteração do gradiente de pressão. O gradiente de pressão é um parâmetro essencial para o projeto de equipamentos
nas indústrias e por isso a sua predição deve ser obtida de forma rápida e confiável. Diversos estudos presentes na literatura avaliam o 
gradiente de pressão para escoamento monofásico em tubos lisos, contudo existe uma carência de estudos para escoamento bifásico em tubos corrugados.  
Levando em consideração que o escoamento de fluidos nas linhas de produção da indústria de petróleo e gás é bifásico,
líquido e gás escoam juntos pela tubulação corrugada,
e que suas linhas de produção possuem diâmetros diferentes entre o poço de extração e a plataforma de 
produção este trabalho tem como objetivo principal analisar experimentalmente a influência do diâmetro interno de um tubo corrugado do tipo d no gradiente de
pressão de um escoamento bifásico horizontal ar-água. Neste estudo serão utilizados três diferentes diâmetros internos de tubulação
(26 mm, 40 mm e 50 mm). Uma bancada experimental contendo os diferentes diâmetros de tubulação corrugada será montado nas instalações do NUEM/UTFPR. 
Como objetivo secundário este trabalho visa propor uma nova proposta para o fator multiplicador de Lochkart-Martinelli.










\palavrasChave{Gradiente de pressão bifásico, tubos corrugados, diâmetro interno, fator multiplicador de Lochkart-Martinelli}