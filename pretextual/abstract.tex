The oil and gas companies uses flexible pipes, also known as corrugated pipes, in offshore production lines 
due the structural characteristics of low bending stiffness and high axial stiffness.
Corrugated pipes are made of several layers of materials where the inner layer has cavities that are 
spaced apart by a constant distance. The cavities of the corrugated pipes are responsible for flexibility that is the main characteristic of 
the corrugated pipes. The presence of cavities in the pipes can provide the dynamic changes in the flow, by example the pressure drop vary. The pressure drop is a essential project requirement for equipments design in industries and a prediction must
be obtained quickly and reliably. Several studies evaluate the pressure drop for single-phase flow in smooth 
pipes. However, there is a lack of studies for two-phase flow in corrugated pipes. The fluid flows in the oil and gas production lines are two-phase flow, liquid and gas flows together through the corrugated pipes, and there is a differece of diameter along the production lines. For that purpose this study will investigate the influence of the inner diameter on pressure drop of a horizontal air-water flow in a d-type corrugated pipe.
For this study will be used three different inner diameters (26 mm, 36 mm and 46 mm).
An experimental flowloop will be assembled by NUEM in UTFPR facilities. This work will propose a new correlation for 
two-phase multiplier of Lockhart-Martinelli.

%In spite of the fact that





\keyWords{Two-phase pressure drop, corrugated pipes, inner diameter, multiplier of Lockhart-Martinelli}