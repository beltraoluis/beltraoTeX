%%%%%%%%%%%%%%%%%%%%%%%%%%%%%%%%%%%%%%%%%%%
% novos comandos
%%%%%%%%%%%%%%%%%%%%%%%%%%%%%%%%%%%%%%%%%%%

% citar figura
\newcommand{\citaeq}[1]{
	Equação\ ~\ref{#1}
}

% citar figura
\newcommand{\citafig}[1]{
	Figura\ ~\ref{#1}
}

% citar tabela
\newcommand{\citatab}[1]{
	Tabela\ ~\ref{#1}
}

% citar quadro
\newcommand{\citaquad}[1]{
	Quadro\ ~\ref{#1}
}

% citar item
\newcommand{\citaitem}[1]{
	item\ ~\ref{#1}
}

% Comando para inserir figuras
\newcommand{\figura}[4]{
	\begin{figure}[H]
		\centering
		\includegraphics[scale=#4]{imagens/#1}
		\caption{#2}
		\label{#3}
	\end{figure} 
}

% comando para inserir novo capitulo
\newcommand{\incluirCapitulo}[1]{
	\input{capitulos/#1}
	\newpage
}

% comando para inserir novo capitulo para relatotorio
\newcommand{\incluirCapituloRelatorio}[1]{
	\input{capitulos/#1}
}

%Comando para inserir tabelas
\newcommand{\tabela}[4]{
	\begin{table}[H]
		\renewcommand{\arraystretch}{1.3}
		\centering
		\caption{#1}
		\label{#4} 
		\begin{tabular}{#2}
			#3
		\end{tabular}
	\end{table}
}


\newcommand{\seculo}[1]{
	século~\MakeUppercase{\romannumeral #1}
}

\newcommand{\fonteautor}{
	Fonte[do autor]
}

\newcommand{\btlnota}[1]{
	\textcolor{red}{\textsc{#1}}
}

\newcommand{\cabecalhoPB}[3]{
	\noindent
	\begin{minipage}[c][2.5cm][c]{2.7cm} % a primeira minipágina tem uma altura de 1.5cm e uma largura de 3cm.
		\includegraphics[height=2.2cm]{imagens/BrasaoDasArmasBrasil_PretoBranco.png}
	\end{minipage}
	\begin{minipage}[c][2.5cm][c]{9cm} % esta é a segunda minipágina que ficará no centro.
		\small
		Ministério da Educação \leading{8pt} \\
		\textbf{Universidade Tecnológica Federal do Paraná}  \leading{8pt} \\
		#1 \leading{8pt} \\
		#2 \leading{8pt} \\
		#3
	\end{minipage}
	\begin{minipage}[c][2.5cm][c]{3cm} % a primeira minipágina tem uma altura de 1.5cm e uma largura de 3cm.
		\includegraphics[width=3cm]{imagens/UTFPR.png}
	\end{minipage}
	\vspace{0.5cm}
}

\newcommand{\cabecalho}[3]{
	\noindent
	\begin{minipage}[c][2.5cm][c]{2.7cm} % a primeira minipágina tem uma altura de 1.5cm e uma largura de 3cm.
		\includegraphics[height=2.2cm]{imagens/BrasaoDasArmasBrasil.png}
	\end{minipage}
	\begin{minipage}[c][2.5cm][c]{9cm} % esta é a segunda minipágina que ficará no centro.
		\small
		Ministério da Educação \leading{8pt} \\
		\textbf{Universidade Tecnológica Federal do Paraná}  \leading{8pt} \\
		#1 \leading{8pt} \\
		#2 \leading{8pt} \\
		#3
	\end{minipage}
	\begin{minipage}[c][2.5cm][c]{3cm} % a primeira minipágina tem uma altura de 1.5cm e uma largura de 3cm.
		\includegraphics[width=3cm]{imagens/UTFPR.png}
	\end{minipage}
	\vspace{0.5cm}
}

\newcommand{\micro}{\textbf{$\upmu$}}

\newcommand{\ohm}{\textbf{$\Upomega$}}

\newcommand{\degree}{\textbf{$^\circ$}}
\newcommand{\citerigo}[1]{\citeauthoronline{#1} (\citeyear{#1})}


