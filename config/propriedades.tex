%%%%%%%%%%%%%%%%%%%%%%%%%%%%%%%%%%%%%%%%%%%%%%%%%%%%%%%%%%%%%%%%%%%%%%%%%%%%%%%%
% Modelo de trabalho segundo as normas da utfprtex
% autor: Luís Henrique Beltrão Santana
% ano: 2015
% originalmenten testado e desenvolvido no linux com o pacote texlive usando o 
% comando PDFLatex usando o editor kile no padrão Unicode UTF-8
%%%%%%%%%%%%%%%%%%%%%%%%%%%%%%%%%%%%%%%%%%%%%%%%%%%%%%%%%%%%%%%%%%%%%%%%%%%%%%%%
\instituicao{Universidade Tecnológica Federal do Paraná}
\departamento{Programa de Pós-Graduação em Engenharia Mecânica e de Materiais}
\curso{Núcleo de Escoamentos Multifásicos}
\autor{Ana Luiza Beltrão Santana }
\titulo{Análise experimental do gradiente de pressão de um escoamento bifásico horizontal ar-água em tubos corrugados}
\tipotrabalho{Projeto de Dissertação}
\data{Setembro 2017}
\local{Curitiba}
\btorientador{Moisés A. Marcelino Neto}
\btcoorientador{Rigoberto E. M. Morales}
\preambulo{\imprimirtipotrabalho \ apresentado ao  \imprimirdepartamento, como 
requisito parcial para a obtenção do Título de Mestre em Engenharia Mecânica na área de concentração em Engenharia Térmica, na 
\imprimirinstituicao.}